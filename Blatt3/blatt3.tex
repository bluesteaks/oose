\documentclass[]{article}
\usepackage[utf8]{inputenc}
\usepackage{listings}


\title{Objektorientierte Softwareentwicklung Blatt 3}
\author{Gruppe I9}
\date{}
\setlength{\parindent}{0pt}

\begin{document}

\maketitle \section*{Aufgabe 1}\textbf{(a)}
Teilausdrücke:\\
(i) $3+4$\quad\quad alles int, keine Typkonversion, Ergebnis int $7$\\
(ii) $(3+4)*2$\quad\quad alles int, keine Typkonversion, Ergebnis int $14$\\
(iii) $(3+4)*2/3$\quad\quad alles int, keine Typkonversion, Ergebnis int $4$\\
(iv) $1+4.0$\quad\quad int $1$ und double $4.0$, int wird zu double konvertiert, Ergebnis double $5.0$\\
(v) $((3+4)*2/3)/(1+4.0)$\quad\quad int $((3+4)*2/3)$, double $(1+4.0)$, int wird zu double konvertiert, Ergebnis double $0.8$\\\\

\textbf{(b)}
Teilausdrücke:\\
(i) $4*2$\quad\quad alles int, keine Typkonversion, Ergebnis int $8$\\
(ii) $(4*2)/9)$\quad\quad alles int, keine Typkonversion, Ergebnis int $0$\\
(iii) $((4*2)/9)+3$\quad\quad alles int, keine Typkonversion, Ergebnis int $3$\\
(iv) $(1.5+2.5)$\quad\quad alles double, keine Typkonversion, Ergebnis double $4.0$\\
(v) $(((4*2)/9)+3)/(1.5+2.5)$\quad\quad Zähler int, Nenner double, int wird zu double konvertiert, Ergebnis double $0.75$

\textbf{(c)}
Teilausdrücke:\\(1.5-0.5)*((13/2+5)/2)
(i) $1.5-0.5$\quad\quad alles double, keine Typkonversion, Ergebnis double $1.0$\\
(ii) $13/2$\quad\quad alles int, keine Typkonversion, Ergebnis int $6$\\
(iii) $13/2+5$\quad\quad alles int, keine Typkonversion, Ergebnis int $11$\\
(iv) $(13/2+5)/2$\quad\quad alles int, keine Typkonversion, Ergebnis int $5$\\
(v) $(1.5-0.5)*((13/2+5)/2)$\quad\quad erster Faktor double, zweiter int, int wird zu double konvertiert, Ergebnis double $5.0$

\maketitle \section*{Aufgabe 2}
Code:
\begin{lstlisting}[frame=single]
if(6.5f==6.5)
	System.out.println("6.5f==6.5");
if(6.4f==6.4)
	System.out.println("6.4f==6.4");
\end{lstlisting}

Output:
\begin{lstlisting}[frame=single]
6.5f==6.5
\end{lstlisting}

Grund: In Java float $6.5f$ ist float, $6.5$ ist double. Da float eine 32bit Darstellung ist und double 64bit, haben wir für double eine genauere Darstellung von Zahlen, die keine exakte Binärdarstellung haben. Die Binärdarstellung von ist $110.1$, also sind sie in float und double gleich. Die Binärdarstellung von $6.4$ ist $110.011001100110011...$. Da mit double eine höhere Genauigkeit, also eine höhere Anzahl von Nachkommastellen möglich ist, sind $6.4f$ und $6.4$ nicht gleich.

\maketitle \section*{Aufgabe 3}
\textbf{(a)}
Implizite Konversionen:\\
(i) von byte nach short, int, long, float, double\\
(ii) von short nach int, long, float, double\\
(iii) von char nach int, long, float, double\\
(iv) von int nach long, float, double\\
(v) von long nach float, double\\
(vi) von float nach double\\\\

\textbf{(b)}
Verlust kann Auftreten bei der Konversion von int bzw long nach float, und von long nach double. Der Grund ist, dass int bzw. long $32$ bit bzw. $64$ bit sind, aber float eine $23$ bit Mantisse hat, das heißt bei allen Integern bzw. Longs größer als $2^{23}$ werden die niedrigwertigsten bits abgeschnitten.
Analog hat double eine $52$ bit Mantisse, also werden bei allen Longs größer als $2^{52}$ die niedrigwertigsten bits abgeschnitten.\\\\

\textbf{(c)}
Seien $A$ und $B$ zwei Typen, sein $a$ von Typ $A$. Wir konvertieren $a$ zu Typ $B$ durch $(B) a$.
Probleme gibt es z.B. wenn $A$ Supertyp von $B$ ist, denn dann kann konvertieren von Typ $A$ nach $B$ Fehler erzeugen: \begin{lstlisting}[frame=single]
public class Animal {...}
public class Dog extends Animal {...}

Animal animal = new Animal();
Dog dog = (Dog) animal;
\end{lstlisting}

wird einen Fehler werfen, denn nicht jedes Tier ist ein Hund.
\\\\
\maketitle \section*{Aufgabe 4}
\textbf{(a)}
\begin{lstlisting}[frame=single]
if(key==1){I();}
else if(key==3){J();K();}
else if(key==4){K();}
else if(key==5 || key==6){L();}
else if(key==7){M();N();}
else {N();}
\end{lstlisting}

\textbf{(b)}
\begin{lstlisting}[frame=single]
1: int i = -1;			//1 Mal
2: out: while (true) { 		//5 Mal
3: 	i++; 			//5 Mal
4: 	int j = -i; 		//5 Mal
5: 	if (i > 6)		//5 Mal
6:	break;			//1 Mal
7: 	if (j < -3)		//4 Mal
8: 	continue;		//2 Mal
9: 	i += 1;			//4 Mal
10: 	in: while (i == -1) {	//4 Mal
11: 	i = -1;			//0 Mal
12: 	break out;		//0 Mal
	}
}
\end{lstlisting}
\end{document}
